\chapter{Diskussion der Ergebnisse}
\label{cha:discussion}
Die Arbeit verlief zu großen Teilen sehr gut. Auf einem lokalen System läuft Sie reibungslos und erfüllt ihre Aufgaben. Dies ist auf dem Server nicht der Fall. Das Deploy ist gescheitert und somit auch die Implementierung. Sollte ein Deploy noch nachträglich umgesetzt werden, so handelt es sich hierbei um ein funktionierende Inventursystem für das Smarthomelab der Hochschule Furtwangen.



\section{Fazit}
Für mich ist das Projekt ein Erfolg. Es hat mich in meinem Wissen weitergebracht und ich habe eine Anwendung umgesetzt, welche alle Phasen durchlaufen ist. Schade ist es, das es am Ende an dem Deploy gescheitert ist. Aber aus diesen Fehlern habe ich gelernt. Ich hätte mit einem Zwischenergebnis die Tücken eines Deploys feststellen können und wäre darauf besser Vorbereitet. Auch wenn die Konzeptionszeit sehr gut verlief, würde ich diese in einem zweiten Anlauf verkürzen. Nichts desto weniger bin ich sehr zufrieden mit dem Ergebnis.


\section{Ausblick}
Auf dieser Arbeit lässt sich aufbauen. So wurde eine \ac{REST}-Schnittstelle entwickelt. Hier könnte man Ansetzen und einen Barcodescanner entwickeln welcher auf diese Schnittstelle zugreift. Dabei spielt es noch nicht einmal eine Rolle um welche Programmiersprache es geht, solange sie REST-Befehle senden kann. Auch könnte mit einer Schnittstelle zu OpenHAB die Status der einzelnen Items erweitert werden. So das man in der Tabelle schon sehen kann welches Item in Betrieb ist. Als Beispiel könnte man zeigen, welche Lampen in welcher Farbe leuchten.