\chapter{Konzeption}
\label{cha:konzeption}
Bei der Konzeption wurde der aktuelle Stand des bestehenden Webauftritts betrachtet und analysiert. Im weiteren Schritt wurden Erweiterungen für die Website geplant. Im nächsten großen Schritt erfolgte eine Analyse aktuell bestehender Software, welche Gegenstände verwaltet.


\section{Analyse des aktuellen Webauftritts}
\label{webauftritt}
Der Webauftritt, erreichbar im Netzwerk der Hochschule Furtwangen\url{http://web.smarthome.hs-furtwangen.de/}, des \ac{SHL} Labors, wird mit der Hilfe des \ac{CMS} WordPress verwaltet. WordPress eigne sich hier, weil es mittels \ac{UX} so gestaltet wurde, das es sehr leicht zu erlernen ist. Unerfahrene Benutzer könne so schnell eigene Inhalte einpflegen.

\begin{figure}[bh]
	\centering
	\includegraphics[scale=0.23]{content/pictures/shlwebsit.jpg}
	% bosch_iot_poll.png: 0x0 pixel, 300dpi, 0.00x0.00 cm, bb=
	\caption{Die Startseite des Webauftritts}
	\label{fig:web}
\end{figure}

Die Website ist mit dem, von WordPress eigenem Designe Twentyseventeen gestaltet. Auf der Startseite sieh man ein großes Headerbild. Darunter Folgt eine Navigationsleiste, welche die Links bei einem Besuch einer Unterseite in einem Lila Farbton darstellt. Die Navigation umfasst die Seiten:

\begin{itemize}
	\item Home
	
	\item The Lab 
	
	\item Components 
	
	\item Panels
	
	\item Use Cases 
	
	\item Info Terminal
	
	\item 3D Model
	
	\item Live Stream
	
	
	
\end{itemize}

Weiter folgt eine kurze Beschreibung über das Labor. Der nächste Abschnitt der Startseite stellt das Team vor. Im Anschluss sieht man noch ein Video, welches das Labor vorstellt, ein Kontaktforumlar und die Addresse mit einer Google Maps Karte. Am untersten Ende ist ein Footer, welche eine Copyright und einen Link zum Impressum enthält.

Die Unterseite \glqq The Lab\grqq stellt das Labor mit Grundrissen etwas detailreicher vor. \glqq Components\grqq stellt wenige, aber wichtige Geräte vor. In \glqq Panels\grqq werden, die mit Sensoren und Geräte versehenen, Wände in den Räumen beschrieben. Eine Auswahl von umgesetzten Anwendungsfällen werden in der Unterseite \glqq Use Cases\grqq beschrieben. Für das Infoterminal gibt es ebenfalls eine Unterseite, welche automatisiert eine Präsentation über das Labor abspielt. Unter dem Punkt \glqq 3D Model\grqq, befindet sich eine 3D Ansicht des Labors, welches sich in einer Ego- und Vogelperspektive betrachten lässt. Auch können hier vereinzelt Geräte bedient werden. Abgeschlossen wird die Navigation mit einer Unterseite, welche einen Video Livestream zeigen kann.

 

\subsection{Corporate Design}
Mit den aktuellen Farben entspricht der Webauftritt des Labors nicht dem \ac{CD} der Hochschule Furtwangen. So müssen die Lila Farben durch die Farbe Grün mit dem Hexwert 83b62d geändert werden. Dies ist wichtig um einen Wiedererkennungswert der Hochschule darzustellen. Die Navigation stellte hier einen größeren Bruch dar. Um die Seite vollständig im \ac{CD} der Hochschule zu haben, ist ein Expertengespräch nötig.



\subsection{Erweiterung des Webauftritts}
Neben dem \acf{CD} gab es noch weitere Punkte, welche den Auftritt noch Informativer und attraktiver gestalten konnten. So stehen im Labor die einzelnen Räume: Küche, Bad, Multimediaraum, IoT-Raum und der Arbeitsbereich stark im Vordergrund. Diese wurden bisher nur sehr dürftig auf der Homepage erwähnt. Daher wurde für die Räume eigene Unterseiten angelegt. Um auf \acf{UX} zu achten, wurde mittels des WordPress-Plugin Draw Attention  eine Interaktive Karte erzeugt. Bei einem Klick auf einen Raum, erhält der User mehr Informationen. \autocite{WPDrawAttention.}

\begin{figure}[bh]
	\centering
	\includegraphics[scale=0.35]{content/pictures/room.jpg}
	% bosch_iot_poll.png: 0x0 pixel, 300dpi, 0.00x0.00 cm, bb=
	\caption{Raumkarte}
	\label{fig:room}
\end{figure}

Jeder Raum hat seine eigene Unterseite, welche die Aufgaben des Raumes und dessen Ausstattung beschreibt.
\newpage
Die Netzwerkarchitektur im Labor ist sehr komplex. Wenn man nicht gerade mit am Aufbau dieser Architektur gearbeitet hat, kann schnell der Überblick verloren werden. Eine Übersichtskarte (siehe Abbildung \ref{fig:network}), auf welcher man sieht wo sich Server, Router und Gateways befinden, kann helfen um herauszufinden, wo bei Problemen mal nachgesehen werden sollte. Diese findet man in der Navigation unter dem neuen Punkt Räume. Neben der Karte gibt es zu jedem Raum eine eigene Unterseite, wo auf dessen Netzwerkeigenschaften genauer eingegangen wird. Die Karte wurde mit Adobe Illustrator CC 2017 erzeugt. Dabei wurde, um die Übersichtlichkeit nicht zu gefährden, sehr auf Schlichtheit geachtet. Nur das nötigste wurde eingezeichnet. Neben der Einfachheit halten sich die Farben Grün und Weiß an das \ac{CD} der Hochschule Furtwangen.

\begin{figure}[bh]
	\centering
	\includegraphics[scale=0.35]{content/pictures/network.png}
	% bosch_iot_poll.png: 0x0 pixel, 300dpi, 0.00x0.00 cm, bb=
	\caption{Netzwerkkarte}
	\label{fig:network}
\end{figure}

\section{Konzeption des Inventarisierungssystem}
\label{konzept:inventar}
Das Inventarisierungssystem ist das Herzstück dieser Masterthesis. Mit ihm sollen alle Gegenstände verwaltet werden können und dazu noch sehr einfach und übersichtlich. Dabei spielt \acf{UX} und damit auch das \acf{UI} eine sehr große Rolle. Um einen Überblick zu bekommen was der aktuelle Markt zu bieten hat, müssen Vergleichbare Verwaltungssysteme(siehe Kapitel \ref{konzept:vergleich}) analysiert werden. Damit eine reibungslose Programmierung erfolgen kann müssen Konzepte für die Architektur der Anwendung ausgearbeitet werden.


\section{Vergleichbare Inventarisierungssysteme}
\label{konzept:vergleich}

Sucht man nach Inventarisierungssysteme stößt man immer wieder auf Software für Webshops. Mit ihnen hat man sehr häufig einen riesigen Umfang an Funktionen, mit welchen man nicht nur sein Inventar, sondern auch seine Verkäufe Verwalten kann.

\subsection{Magento}
\label{konzeption:magento}
Magento\footnote{https://magento.com} ist eine Openen-Source-E-Commerce Plattform und steht unter der Open Software License\footnote{https://opensource.org/licenses/osl-3.0.php}. Umgesetzt wurde es mit dem \acp{PHP} Framework Zend und lässt sich durch zahlreiche Plug-Ins erweitern.\autocite{.2018}

\begin{figure}[bh]
	\centering
	\includegraphics[scale=0.35]{content/pictures/magento.png}
	% bosch_iot_poll.png: 0x0 pixel, 300dpi, 0.00x0.00 cm, bb=
	\caption{Magento Itemverwaltung}
	\label{fig:magento}
\end{figure}

\subsubsection{Vorteile von Magento}
Magento bringt zahlreiche Verwaltungsoptionen. Man kann bis in das kleinste Detail einen Webshop konfigurieren und verwalten. Eine Rollenverteilung der User wird ebenfalls geboten. Jeder Gegenstand kann mit vorgefertigten und eigen angelegten Eigenschaften versehen werden. Das \ac{UI} von Magento ist ähnlich wie bei WordPress sehr benutzerfreundlich. Mit ein wenig Erfahrung findet man sich schnell zurecht. \autocite{Koch.2012}

\subsubsection{Nachteile von Magento}
In seiner größe, liegt der Nachteil des Frameworks. Es bietet zu viele Funktionen, welche nicht für ein Inventursystem in einer Laborumgebung benötigt werden. Damit belegt es unnötig Speicherplatz und lädt Dateien, welche nicht gebraucht werden. Auch wenn die Lernkurve sehr flach ist, muss eine gewisse Einarbeitung doch erfolgen. Auch ist mit der aktuellen Version von Magento eine Aktualisierung der \ac{PHP} Version nötig. \autocite{Koch.2012}

\subsection{WeeComerce}
Anders als bei Magento (siehe Kapitel \ref{konzeption:magento}) steht das WordPress-Plugin, welches auch als E-Commerce Software dient unter der \ac{GNU} Lizenz. 

\subsubsection{Vorteile von WeeComerce}
Bei Kostenlose WordPress-Plugin ist kostenlos und sehr einfach zu installieren. Mit einem klick ist es aus dem Plugin Bereich gewählt und kann mit einem Installationsassistenten den eigenen Bedürfnissen nach angepasst werden. Ein erfahrener WordPress-Nutzer hat eine deutlich niedrigere Lernkurve, als bei Magento \autocite{WeeComerce.2018}

\subsubsection{Nachteile von WeeComerce}
Die Nachteile sind ähnlich wie bei Magento. Es beinhaltet zu viele Funktionen, welche nicht gebraucht werden. Außerdem ist es sehr schwer möglich es nicht als Webshop, sondern als Verwaltungssystem zu nutzen. WeeComerce gibt hier sehr starke richtlienen vor. So muss man schon im Installationsassisten sich gedanken über Maße, Bezahlmethoden und Versand machen. Was für ein Inventarisierungssystem nicht nötig ist. Mit WeeComerce ist man stark an WordPress gebunden. Solle man das \ac{CMS} wechseln wollen, ist ein mitnehmen der Anwendung nicht möglich.

\subsection{Entscheidung zur eigenen Anwendung}
Die Entscheidung eine eigene Anwendung zu schreiben wurde schon früh getroffen. Es sollte eine Anwendung zur Verfügung gestellt werden, welche nicht mit Funktionen überladen ist. Die Lernkurve muss für jeden Anwender möglichst flach gehalten sein. Der Aufwand Magento oder WeeComerce in das Hochschul \ac{CD} zu bringen, kann ein großer Aufwand sein. Daher lohnt es sich die Anwendung selbst zu schreiben. So bleibt sie Konfigurierbar und es werden nur \ac{PHP} Kenntnisse und keine spezielen WordPress oder Magento-Kenntnisse vorausgesetzt. Bei einem Wechsel zu einem anderen \ac{CMS} kann die Anwendung ebenfalls übernommen werden.

\section{Anforderungen}

Mit dem Anforderungsmanagement werden mögliche Features definiert. In der Anforderungsanalyse werden Eigenschaften, Funktionalitäteten und die Qualität an die Software Festgehalten. \autocite{Grande.2014} Es werden funktionale und nicht-funktionale Anforderungen niedergeschrieben. Mit diesen Anforderung entsteht eine Definition und Ziele an die Anwendung.

\begin{itemize}
	\item Das Inventarisierungssystem muss alle Items in einer Laborumgebung aufnehmen. 
	
	\item Dabei soll bekannt sein wie der aktuelle Zustand der jeweiligen Items ist. 
	
	\item Die Datenbank sollen durchsuchbar sein. 
	
	\item Es müssen neue Items angelegt und alte gelöscht werden können. 
	
	\item Es soll für jeden Benutzer einen persönlichen Bereich geben. 
	
	\item Es soll Verschiedene Rollen für die Administration und User geben.
	
	\item Jeder Benutzer muss seine Persönlichen Daten verändern können.
	
	\item Benutzer können über den Administrator oder einem dazu befugten Mitarbeiter Items leihen.
	
	\item Benutzer sollen an die Rückgabe der geliehenen Items erinnert werden.
	
	
\end{itemize}

Ist eine solche Liste von Anforderungen erstellt worden, erhält man eine gute Übersicht über die zu entwickelnden Teilbereiche.


\section{Datenbank}

\section{MockUps}

\section{User Experience}

\section{Fitts Gesetze}

\section{Wahl der Frameworks}
