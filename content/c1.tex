\chapter{Einleitung}
Das Smarthome-Labor der Hochschule Furtwangen umfasst vier einzelne Räume und ein Arbeitsbereich, in welchem sich Studierende, Mitarbeiterinnen und Mitarbeiter sich auf ihre wichtigen Arbeiten im Labor konzentrieren können. Das noch junge Labor steckt voller Leben. Es wird mit Bachelor-Projekten, Bachelor-Thesen und Master-Thesen immer weiter ausgebaut und die Studierenden lernen praxisnah den Umgang mit Geräten in einer Smarthome-Umgebung. Diese betrifft nicht nur die Unterkunft in welcher Personen leben, sondern auch die Industrie 4.0.

Im Sommersemester 2017 entstand die Masterthesis von Ashiq Mohamed Akbar Ali mit dem Titel "Creating a website with Info-Terminal and Live CCTV Stream for the Smart Home Laboratory at the Hochschule Furtwangen University". Mit dieser Arbeite wurde das Labor mit einer umfangreichen Webpräsenz ausgestattet, welche einen Einblick über die die Vielfalt im Labor gibt. Das Labor wird hierbei vorgestellt und es wird auf einzelne Details eingegangen. \autocite{akbarali}

Es gibt eine Vielzahl an Geräten, Sensoren, Mikrocomputer und Alltagsgegenstände, wie zum Beispiel das Bett, mit welchen die Studierenden experimentieren und einen Betrag für die Wissenschaft und Öffentlichkeit bieten können.




\section{Problemstellung}
\label{sec:Problemstellung}

Durch diese große Anzahl von Geräten, Sensoren und Mikrocomputer entsteht auch eine Unübersichtlichkeit. Zwar wurden alle Gegenstände erfasst und Dokumentiert, jedoch ist das nur in einer Tabelle erfasst worden. Ein Durchsuchen dieser Tabelle kann Aufwendig sein wenn ein bestimmtes Gerät und dessen Status überprüft werden soll. Für ihre Arbeiten müssen sich Studenten auch Geräte reservieren oder gar ausleihen um sie in fremden Umgebungen testen zu können oder um daheim weiter arbeiten zu können. Dies in einer Tabelle zu erfassen ist nicht unmöglich, aber es handelt sich hierbei um einen großen Aufwand. Dieser gestaltet die Arbeit von einer Mitarbeiterin oder einem Mitarbeite, als sehr Aufwendig. Diese Tabelle muss ständig kontrolliert und aktualisiert werden. Auch muss geprüft werden, ob die Leihe für ein Gerät schon verstrichen ist.


\section{Ziel der Arbeit}
\label{sec:ziel}
Ziel ist eine Übersicht in das Umfangreiche Inventar des Smarthome-Labors zu geben. Dabei kann eine Webanwendung helfen, welche in einer Tabelle Alle Geräte, Sensoren und Mikrocomputer Aufnimmt. Diese Tabelle kann mittels Software schnell durchsucht und einfach erweitert werden. Auch soll die Mitarbeiterin oder der Mitarbeiter, wie auch Studierende an das Ende einer Leihfrist erinnert werden. Auch hierbei ist die Software eine Lösung, indem sie Erinnerungsmails verschickt.

