\begin{table}[H]
 \centering
 \caption{Steckbrief: Apache Karaf}
 \begin{framed}
 \begin{tabular}{l|l|l|l}
  \multicolumn{4}{l}{}\\
  \multicolumn{4}{l}{\textbf{Apache Karaf}}\\
  \multicolumn{4}{l}{}\\
  \toprule
  %Nr. &  Anforderung & Wertebereich \\
  %\midrule
  \multicolumn{2}{l|}{Webseite des Projektes} & \multicolumn{2}{p{8cm}}{\url{https://karaf.apache.org/}}\\
  
  \multicolumn{2}{l|}{} & \multicolumn{2}{l}{}\\
  
  \multicolumn{2}{l|}{Beginn der Entwicklung} & \multicolumn{2}{p{8cm}}{Mitte 2010}\\
  
  \multicolumn{2}{l|}{} & \multicolumn{2}{l}{}\\
  
  \multicolumn{2}{l|}{Stand Entwicklung} & \multicolumn{2}{p{8cm}}{Aktuelle Version: 4.1.2 vom 04.08.2017} \\
  
  \multicolumn{2}{l|}{} & \multicolumn{2}{l}{} \\
  
  \multicolumn{2}{l|}{Beschreibung} &  \multicolumn{2}{p{8cm}}{
  Bei Apache Karaf handelt es sich um einen OSGi-basierten polymorphen Container.
  Innerhalb des Containers können Anwendungen separiert voneinander ausgeführt werden.
  } \\
  
  \multicolumn{2}{l|}{} & \multicolumn{2}{l}{} \\
  
  \multicolumn{2}{l|}{Schwerpunkt} &  \multicolumn{2}{p{8cm}}{
  Der Schwerpunkt liegt im Betrieb von Applikationen und Micro-Services. Apache Karaf unterstützt dafür eine 
  große Anzahl an unterschiedlichen Anwendungsformaten wie OSGi- und Spring-Bundles, sowie \ac{KAR}-, \ac{WAR}- und \ac{WAB}-Anwendungen.
  } \\
  
  \multicolumn{2}{l|}{} & \multicolumn{2}{l}{} \\
  
  \multicolumn{2}{l|}{Besondere Features} & \multicolumn{2}{p{8cm}}{
  Hot-Deployment von verschiedenen Applikationen,
  Provisioning über \ac{OBR}, maven-Repositories, etc.,
  Karaf Features zur Beschreibung von Karaf-Anwendungen } \\
  
  \multicolumn{4}{l}{} \\
  
  Nr. &  \multicolumn{2}{l|}{Kriterium} & Bewertung \\
  \midrule
  1 & \multicolumn{2}{l|}{Sicherheit}
  & 1 \\
  
  2 & \multicolumn{2}{l|}{Zentrales Management}
  & nein \\
  
  3 & \multicolumn{2}{l|}{Distributionen}
  & ja \\
  
  4 & \multicolumn{2}{l|}{Ausfallsicherheit}
  & nein \\
  
  5 & \multicolumn{2}{l|}{Schnittstellen (\ac{API})}
  & ja \\
  
  6 & \multicolumn{2}{l|}{Zielplattform OSGi}
  & ja \\
  
  7 & \multicolumn{2}{l|}{Existierende Community}
  & ja \\
  
  8 & \multicolumn{2}{l|}{Erkennbare Weiterentwicklung}
  & 4 \\
  
  9 & \multicolumn{2}{l|}{Langlebiges Projekt}
  & 3 \\
  
  10 & \multicolumn{2}{l|}{Flexible Wartbarkeit}
  & 1 \\
  
  11 & \multicolumn{2}{l|}{Anzahl Entwickler}
  & 4 \\
  \bottomrule
 \end{tabular}
 \label{tab:rating_karaf}
 \end{framed}
\end{table}

%\footnotetext{Erste Version: \url{https://mvnrepository.com/artifact/org.apache.felix/org.osgi.service.obr/1.0.0}}