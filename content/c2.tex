\chapter{Vorgehen}
In der Vorbereitung wurden schon früh die Werkzeuge für die Umsetzung und Konzeption gewählt, welche das Projekt strukturieren und planen sollten. Dabei kamen Mittel wie Kugelschreiber, Notizbücher, Kalender oder auch die Mindmap zum Einsatz. Mittels Requirement Engeneering wurde eine Anforderungsanalyse erstellt. Um einen übersichtlichen Workflow zu haben, fiel die Entscheidung Scrum als Vorgehensmodell für den künftigen Ablauf zu wählen.



\section{Anforderungen}

Mit dem Anforderungsmanagement werden mögliche Features Defininiert. In der Anforderungsanalyse werden Eigenschaften, Funktionalitäteten und die Qualität an die Software Festgehalten. \autocite{100Minuten} Es werden funktionale und nicht-funktionale Anforderungen niedergeschrieben. Mit diesen Anforderung ensteht eine Definition und Ziele an die Anwendung.

\begin{itemize}
\item Das Inventarisierungssystem muss alle Items in einer Laborumgebung aufnehmen. 

\item Dabei soll bekannt sein wie der aktuelle Zustand der jeweiligen Items ist. 

\item Die Datenbank sollen durchsuchbar sein. 

\item Es müssen neue Items angelegt und alte gelöscht werden können. 

\item Es soll für jeden Benutzer einen persönlichen Bereich geben. 

\item Es soll Verschiedene Rollen für die Administration und User geben.

\item Jeder Benutzer muss seine Persönlichen Daten verändern können.

\item Benutzer können über den Administrator oder einem dazu befugten Mitarbeiter Items leihen.

\item Benutzer sollen an die Rückgabe der geliehenen Items erinnert werden.


\end{itemize}

Ist eine solche Liste von Anforderungen erstellt worden erhält man eine gute Übersicht über die zu entwickelnden Teilbereiche.



\section{Scrum}
\label{scrum}



