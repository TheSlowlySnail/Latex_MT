% ##################################################
% Unterstuetzung fuer die deutsche Sprache
% ##################################################
%\usepackage{ngerman}
\usepackage[ngerman]{babel}
\usepackage[utf8]{inputenc}
\usepackage{fontspec}
\usepackage{textcomp}
\usepackage{lmodern}

% ##################################################
% Aufzählung mit Prefix \begin{enumerate}[label={Nr. \arabic*}, leftmargin=*, labelindent=1em]
% ##################################################
\usepackage{calc}
\usepackage{enumitem}
\setlist[description]{
  font={\normalfont\itshape\sffamily},
  before={\normalfont\sffamily}
}


% ##################################################
% Dokumentvariablen
% ##################################################

% Persoenliche Daten
\newcommand{\docNachname}{Hübner}
\newcommand{\docVorname}{Simon}
\newcommand{\docStrasse}{Berliner Straße 19}
\newcommand{\docOrt}{Furtwangen}
\newcommand{\docPlz}{78120}
\newcommand{\docEmail}{simon.huebner@hs-furtwangen.de}
\newcommand{\docMatrikelnummer}{251906}

% Dokumentdaten
%\newcommand{\docTitle}{Einsatz von Software Deployment Werkzeugen für OSGi-basierte Gateways am Beispiel von OpenMUC}
%\newcommand{\docTitle}{Evaluation von Open-Source Werkzeugen zur Softwareverteilung für das OSGi-basierte Gateway OpenMUC}
\newcommand{\docTitle}{Evaluation von Open-Source-Werkzeugen zur Softwareverteilung für OSGi-basierte IoT-Gateways am Beispiel von OpenMUC}
\newcommand{\docUntertitle}{} % Kein Untertitel SW Deploy, IoT, ..., Bingo!
% \newcommand{\docUntertitle}{UNTERTITEL}
% Arten der Arbeit: Bachelorthesis, Masterthesis, Seminararbeit, Diplomarbeit
\newcommand{\docArtDerArbeit}{Masterarbeit}
%Studiengaenge: Allgemeine Informatik Bachelor, Computer Networking Bachelor,
% Software-Produktmanagement Bachelor, Advanced Computer Scinece Master
\newcommand{\docStudiengang}{Mobile Systeme (MOS)}
\newcommand{\docAbgabedatum}{\today}
\newcommand{\docErsterReferent}{Prof. Dr. Lothar Piepmeyer}
\newcommand{\docZweiterReferent}{Prof. Dr. Elmar Cochlovius} % Wenn es nur einen Betreuer gibt
% \newcommand{\docZweiterReferent}{-}

% ##################################################
% Allgemeine Pakete
% ##################################################

% Abbildungen einbinden
\usepackage{graphicx}
\usepackage{float}
\usepackage{framed}

% Farben
\usepackage{color}
\usepackage[usenames,dvipsnames,svgnames,table]{xcolor}

% Maskierung von URLs und Dateipfaden
\usepackage[hyphens,spaces,obeyspaces]{url}

% setzt Tabellen auch dorthin, wo sie erstellt wereden!
\usepackage[section]{placeins}
% Deutsche Anfuehrungszeichen
\usepackage[babel, german=quotes]{csquotes}

% Pakte zur Index-Erstellung (Schlagwortverzeichnis)
% \usepackage{index}
% \makeindex

% Ipsum Lorem
% Paket wird nur für das Beispiel gebraucht und kann gelöscht werden
\usepackage{lipsum}

% ##################################################
% Seitenformatierung
% ##################################################
\usepackage[
	portrait,
	bindingoffset=1.5cm,
	inner=2.5cm,
	outer=2.5cm,
	top=3cm,
	bottom=2cm,
	%showframe,
	%includeheadfoot
	]{geometry}

% ##################################################
% Kopf- und Fusszeile
% ##################################################

\usepackage{fancyhdr}

\pagestyle{fancy}
\fancyhf{}
%\fancyhead[EL,OR]{\sffamily\thepage}
%\fancyhead[ER,OL]{\sffamily\leftmark}
\fancyhead[EL,OR]{\sffamily\bfseries\thepage}
\fancyhead[ER,OL]{\sffamily\bfseries\leftmark}

\fancypagestyle{headings}{}

\fancypagestyle{plain}{}

\fancypagestyle{empty}{
  \fancyhf{}
  \renewcommand{\headrulewidth}{0pt}
}

%Kein "Kapitel # NAME" in der Kopfzeile
\renewcommand{\chaptermark}[1]{
	\markboth{#1}{}
   	\markboth{\thechapter.\ #1}{}
}

% ##################################################
% Schriften
% ##################################################

% Stdandardschrift festlegen
\renewcommand{\familydefault}{\sfdefault}

% Standard Zeilenabstand: 1,5 zeilig
\usepackage{setspace}
%\onehalfspacing 

% Schriftgroessen festlegen
\addtokomafont{chapter}{\sffamily\large\bfseries} 
\addtokomafont{section}{\sffamily\normalsize\bfseries} 
\addtokomafont{subsection}{\sffamily\normalsize\mdseries} 
\addtokomafont{caption}{\sffamily\normalsize\mdseries} 
% \addtokomafont{paragraph}{\sffamily\normalsize\mdseries}

% ##################################################
% Inhaltsverzeichnis / Allgemeine Verzeichniseinstellungen
% ##################################################

\usepackage{tocloft}

% Punkte auch bei Kapiteln
\renewcommand{\cftchapdotsep}{3}
\renewcommand{\cftdotsep}{3}

% Schriftart und -groesse im Inhaltsverzeichnis anpassen
\renewcommand{\cftchapfont}{\sffamily\normalsize}
\renewcommand{\cftsecfont}{\sffamily\normalsize}
\renewcommand{\cftsubsecfont}{\sffamily\normalsize}
\renewcommand{\cftchappagefont}{\sffamily\normalsize}
\renewcommand{\cftsecpagefont}{\sffamily\normalsize}
\renewcommand{\cftsubsecpagefont}{\sffamily\normalsize}

%Zeilenabstand in den Verzeichnissen einstellen
\setlength{\cftparskip}{.5\baselineskip}
\setlength{\cftbeforechapskip}{.1\baselineskip}

% ##################################################
% Abbildungsverzeichnis und Abbildungen
% ##################################################

\usepackage{caption}

\usepackage{wrapfig}
\usepackage{framed}

% Nummerierung von Abbildungen
\renewcommand{\thefigure}{\arabic{figure}}
\usepackage{chngcntr}
\counterwithout{figure}{chapter}

% Abbildungsverzeichnis anpassen
\renewcommand{\cftfigpresnum}{Abbildung }
\renewcommand{\cftfigaftersnum}{:}

% Breite des Nummerierungsbereiches [Abbildung 1:]
\newlength{\figureLength}
\settowidth{\figureLength}{\bfseries\cftfigpresnum\cftfigaftersnum}
%\setlength{\cftfignumwidth}{\figureLength}
\setlength{\cftfignumwidth}{2.6cm}
\setlength{\cftfigindent}{0cm}

% Schriftart anpassen
\renewcommand\cftfigfont{\sffamily}
\renewcommand\cftfigpagefont{\sffamily}

\usepackage{subcaption}

% ##################################################
% Tabellenverzeichnis und Tabellen
% ##################################################

% Nummerierung von Tabellen
\renewcommand{\thetable}{\arabic{table}}
\counterwithout{table}{chapter}
\usepackage{tabu}
\usepackage{booktabs}

% Fußnoten in Tabellen
\usepackage{threeparttable} 
\usepackage{multirow}

\usepackage[table]{xcolor}
\usepackage{etoolbox}
\AtBeginEnvironment{tabular}{\footnotesize}
\AtBeginEnvironment{longtable}{\footnotesize}

% Dadurch können mit x{size} rechtsbündige und mit z{size} zentrierte Zellen
% mit fester Größe size erstellt werden.
\newcolumntype{y}[1]{>{\raggedright\arraybackslash\hspace{0pt}}p{#1}}
\newcolumntype{x}[1]{>{\raggedleft\arraybackslash\hspace{0pt}}p{#1}}
\newcolumntype{z}[1]{>{\centering\arraybackslash\hspace{0pt}}p{#1}}

% Tabellenverzeichnis anpassen
\renewcommand{\cfttabpresnum}{Tabelle }
\renewcommand{\cfttabaftersnum}{:}

% Breite des Nummerierungsbereiches [Abbildung 1:]
\newlength{\tableLength}
\settowidth{\tableLength}{\bfseries\cfttabpresnum\cfttabaftersnum}
%\setlength{\cfttabnumwidth}{\tableLength}
\setlength{\cfttabnumwidth}{2.0cm}
\setlength{\cfttabindent}{0cm}

%Schriftart anpassen
\renewcommand\cfttabfont{\sffamily}
\renewcommand\cfttabpagefont{\sffamily}

% Unterdrueckung von vertikalen Linien
\usepackage{booktabs}

% ##################################################
% Listings (Quellcode)
% ##################################################

\usepackage{listings}
\definecolor{gray}{rgb}{0.4,0.4,0.4}
\definecolor{darkblue}{rgb}{0.0,0.0,0.6}
\definecolor{cyan}{rgb}{0.0,0.6,0.6}

\lstset{
  basicstyle=\ttfamily,
  %basicstyle=\normalsize,
  columns=fullflexible,
  showstringspaces=false,
  breaklines=true,
  breakautoindent=true,
  numbers=left,
  numberstyle=\tiny,
  numbersep=5pt,
  keywordstyle=\color{blue},
  commentstyle=\color{green},   
  stringstyle=\color{gray},
  linewidth=14.6cm,
  xleftmargin=0.45cm,
}

\lstdefinelanguage{XML}
{
  morestring=[b]",
  morestring=[s]{>}{<},
  morecomment=[s]{<?}{?>},
  stringstyle=\color{black},
  identifierstyle=\color{darkblue},
  keywordstyle=\color{cyan},
  morekeywords={xmlns,version,type}% list your attributes here
}

% \lstset{
% 	language=java,
% 	backgroundcolor=\color{white},
% 	breaklines=true,
% 	prebreak={\carriagereturn},
%  	breakautoindent=true,
%  	numbers=left,
%  	numberstyle=\tiny,
%  	stepnumber=2,
%  	numbersep=5pt,
%  	keywordstyle=\color{blue},
%    	commentstyle=\color{green},   
%    	stringstyle=\color{gray}
% }
  	
% ##################################################
% Theoreme
% ##################################################
  	
% Umgebung fuer Beispiele
\newtheorem{beispiel}{Beispiel}

% Umgebung fuer These
\newtheorem{these}{These}

% Umgebung fuer Definitionen
\newtheorem{definition}{Definition}
  	
% ##################################################
% Literaturverzeichnis
% ##################################################

\usepackage{bibgerm}
%\usepackage[backend=biber, bibencoding=utf8]{biblatex}
%\usepackage{apacite}

% ##################################################
% Abkuerzungsverzeichnis
% ##################################################

%\usepackage[printonlyused]{acronym}
\usepackage{acronym}
%\renewcommand*{\acsfont}[1]{\normalfont\bfseries #1}

% ##################################################
% PDF / Dokumenteninternelinks
% ##################################################

\usepackage[
    unicode=true,
    pdfencoding=unicode,
    colorlinks=false,
    linkcolor=black,
    citecolor=black,
    filecolor=black,
    urlcolor=black,
    bookmarks=true,
    bookmarksopen=true,
    bookmarksopenlevel=3,
    bookmarksnumbered,
    plainpages=false,
    pdfpagelabels=true,
    hyperfootnotes,
    pdftitle ={\docTitle},
    pdfauthor={\docVorname~\docNachname},
    pdfcreator={\docVorname~\docNachname}]{hyperref}