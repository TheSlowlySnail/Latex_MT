\chapter{Einleitung}
\glqq The OSGi\texttrademark{} Alliance was founded in March 1999. Its mission is to create open specifications for 
the network delivery of managed services to local networks and devices. The OSGi organization is
the leading standard for next-generation Internet services to homes, cars, mobile phones, desktops,
small offices, and other environments.\grqq\ \cite[S. 9]{osgi_r6}\\

Der Begriff \acs{OSGi}\footnote{Webseite der OSGi-Alliance: \url{https://www.osgi.org/}} stand früher für \aclu{OSGi}. Heutzutage ist selbst in der Spezifikation nur noch das Akronym gebräuchlich.
Bei \ac{OSGi} muss zwischen der Spezifikation und der konkreten Implementierungen unterschieden werden. Die Spezifikation wird durch die \ac{OSGi}-Alliance erstellt und
beschreibt eine modulare, von der Hardware unabhängige, Softwareplattform. Gegründet wurde die \ac{OSGi}-Alliance von 15 Gründungsmitgliedern im Jahre 1999,
wobei der Fokus damals auf der Spezifikation eines Java-Komponentensystems für Smart Home Gateways lag \cite{heise_osgi}.
Mittlerweile hat sich der Fokus geändert. Die aktuellste Version der Spezifikationen ist derzeit das \ac{OSGi}-Release 6 (R6). 
\glqq Die Spezifikationen umfassen nicht nur den Enterprise-Markt, sondern auch Smart Home, Telematik und Mobile. In Zeiten einer zunehmenden Industrieverflechtung und Anforderungen
an End-To-End-Lösungen wird die Relevanz der \ac{OSGi}-Spezifikationen - von der standardisierten modularen Ausführungsumgebung und vielfältigen Schnittstellen über
eine Geräteabstraktion und eine Entwicklungsumgebung für Anwendungen bis hin zu Cloud-Architekturen für PaaS und SaaS - deutlich\grqq\ \cite{osgi_iot_und_mobile}.
In vielen Bereichen der Informatik, besonders bei kommerziellen Produkten, herrscht eine starke Heterogenität.
\ac{OSGi} unterstützt die Entwicklung von Anwendungen und Frameworks für End-To-End-Lösungen, in Gebieten, in denen eine hohe Flexibilität und Diversität vorherrscht.
Durch den hohen Grad an Modularisierung beim Einsatz von \ac{OSGi} können Teile der Software leicht ausgetauscht, beziehungsweise neue Funktionalitäten einfach implementiert werden.
Dies führt zu einer sehr hohen Anpassungsfähigkeit der \ac{OSGi}-Plattform.\\

Neben der Spezifikation gibt es die konkreten Implementierungen in Form mehrerer \ac{OSGi}-Plattformen. Mehrere kommerzielle Hersteller und verschiedene Open-Source-Projekte bieten
Implementierungen der Plattform an.
Das Ziel dieser Arbeit besteht darin, die Herausforderungen und Probleme der Softwareverteilung für \ac{OSGi}-basierte Gateways zu identifizieren und zu lösen.
Als Grundlage dieser Untersuchung dient das auf \ac{OSGi}-basierende Framework OpenMUC\footnote{Projektwebseite OpenMUC: \url{https://www.openmuc.org/}}. Dabei handelt es sich um ein Gatway zur Überwachung, Protokollierung und Steuerung von vernetzten Systemen.
Genutzt wird das Framework derzeit schwerpunktmäßig im Bereich der intelligenten Energienetze. Die Modularität der Plattform erlaubt aber einen Einsatz in allen \ac{IoT}-Domänen.
Im Weiteren werden verschiedene Werkzeuge der Softwareverteilung für \ac{OSGi}-basierte Anwendungen und Frameworks analysiert und ausgewertet. 
Die vielversprechendste Technologie wird anschließend prototypisch in OpenMUC integriert. 
Mit Hilfe zweier Teststellungen wird die Arbeit evaluiert und weitere Erkenntnis gewonnen.


\section{Problemstellung}
\label{sec:Problemstellung}
Auf der Basis von \ac{OSGi} lassen sich modularisierte Anwendungen entwickeln, welche die stark variierenden Anforderungen, die an \ac{IoT}-Gateways gestellt werden, erfüllen können.
Die Modularisierung erfolgt dabei über sogennante Bundles. Bei einem Bundle handelt es sich um eine erweiterbare und herunterladbare Anwendung \cite{osgi_r6}.
Auch Drittparteien erhalten die Möglichkeit, Bundles zu erstellen, um neue Funktionen in ein Gateway zu integrieren.
Ein Problem dabei ist jedoch die Installation und Wartung eines Bundles. Im Fall von OpenMUC wird das Framework als Archiv, bestehend aus mehreren Bundles und Dateien zur Konfiguration,
bereitgestellt. Änderungen in Teilen z.B. in einem Bundle führen zu einer neuen Version des gesamten Frameworks.
Es ist momentan nicht ohne Weiteres möglich, Updates einzelner Bundles durchzuführen, oder laufende Installationen mit neuen Bundles und Updates zu versorgen.
Dies sind jedoch elementare Anforderungen an ein \ac{IoT}-Gateway.
Vor allem dann, wenn mehr als ein Gateway betrieben wird und die Sensoren und Aktoren von Installation zu Installation stark variieren. 
Wünschenswert ist es, dass die Konfiguration der Gateways und die für die Aufgabe benötigten Bundles an einer zentralen Stelle verwaltet, automatisch verteilt und aktualisiert werden können. 
Die Einführung einer zentralen Management- und Software-Deployment-Komponente führt zu weiteren Anforderungen im Bereich Sicherheit und Verfügbarkeit.

\section{Ziel der Arbeit}
\label{sec:ziel}
Das Ziel der Arbeit besteht darin, eine Komponente zur Verteilung von Software in das OpenMUC-Framework zu integrieren. 
Um dieses Ziel zu erreichen und das in \autoref{sec:Problemstellung} skizzierte Problem zu lösen, müssen Antworten auf folgende Fragen gefunden werden:

\begin{description}
 \item[Forschungsfrage 1:] Welche etablierten Open-Source-Werkzeuge für die Softwareverteilung existieren im Umfeld von \ac{OSGi}?
 \item[Forschungsfrage 2:] Welche Eigenschaften, Vorteile und Nachteile haben die in \textit{Forschungsfrage 1} gefundenen Lösungen?
 \item[Forschungsfrage 3:] Was sind typische und zukünftige Einsatzszenarien für OpenMUC und welche Anforderungen in Bezug auf die Softwareverteilung ergeben sich daraus?
 \item[Forschungsfrage 4:] Welches der in \textit{Forschungsfrage 1 und 2} ermittelten und untersuchten Werkzeuge ist für die Aufgabenstellung von OpenMUC am geeignetsten?
 \item[Forschungsfrage 5:] Welche Anpassungen am OpenMUC-Framework sind notwendig, um das in den vorangegangenen Forschungsfragen evaluierte Werkzeug zu integrieren?
 \item[Forschungsfrage 6:] Erweitert die Integration eines Werkzeuges zur Softwareverteilung die Modularität und die Flexibilität des OpenMUC-Frameworks und erleichtert dies
 den Einsatz in weiteren \ac{IoT}-Domänen?
\end{description}

Um Antworten auf diese Forschungsfragen zu finden, werden zunächst gängige Werkzeuge zur Softwareverteilung untersucht und bewertet.
Im Anschluss daran erfolgt anhand der Erkenntnisse aus \textit{Forschungsfrage 4} eine prototypische Implementierung.
Die erste Implementierung und Integration in OpenMUC erfolgt im Rahmen einer Teststellung aus dem Bereich der intelligenten Energienetze.
Anhand dieser Teststellung sollen erste Erkenntnisse gewonnen werden, wie das Werkzeug in das Framework integriert werden kann und welche Anpassungen dafür notwendig sind.
Die zweite Teststellung findet ihre Anwendung innerhalb der \ac{IoT}-Domäne Smart-Home und dient zur Evaluierung von \textit{Forschungsfrage 6}.
Zu diesem Zweck wird im Rahmen der Teststellung eine typische Routine im Smart-Home-Labor der \ac{HFU} abgebildet und im Vergleich mit dem dort zum Einsatz
kommenden openHAB\footnote{Projektwebseite openHAB: \url{https://www.openhab.org/}} bewertet.
Im folgenden Abschnitt wird der Aufbau der Arbeit und das weitere Vorgehen während der Bearbeitung aufgezeigt.

\section{Aufbau und Vorgehen}
Das Vorgehen zur Beantwortung der Forschungsfragen ist in \autoref{fig:work_composition} zu sehen.
Im Anschluss an dieses Kapitel folgt, ausgehend von der Recherche, ein Kapitel zu den Grundlagen dieser Arbeit.
Dort wird zum einen das Ergebnis der Literaturrecherche vorgestellt und zum anderen die Grundlagen 
der \ac{OSGi}-Plattform, des Softwareverteilungsprozesses und des OpenMUC-Frameworks dargelegt.
Außerdem werden verschiedene Sicherheitsaspekte in Bezug auf die Softwareverteilung erörtert.\\

In \autoref{cha:evaluation} wird die durchgeführte Evaluation der Werkzeuge eingehend beschrieben.
Die zu Beginn recherchierten Werkzeuge werden auf ihre Stärken, Schwächen und Eigenschaften hin untersucht.
Daraufhin wird anhand verschiedener Quellen ein Kriterienkatalog erstellt und die Werkzeuge werden mithilfe dieser Kriterien bewertet.
Abgeschlossen wird \autoref{cha:evaluation} mit der Auswahl des am besten geeigneten Werkzeuges mittels einer Entscheidungsmatrix.

\begin{figure}[ht]
 \centering
 \includegraphics[scale=0.19]{../02.Diagramme/work_composition_v4.png}%width=\textwidth
 \caption{Vorgehen zur Beantwortung der Forschungsfragen}
 \label{fig:work_composition}
\end{figure}


In \autoref{cha:prototype} wird ein Betriebskonzept für die Integration des Werkzeuges erarbeitet und anhand dessen ein erster Prototyp angefertigt.
Dieser Prototyp wird im Rahmen einer Teststellung im Smart-Home-Labor der Hochschule Furtwangen getestet und mit einem etablierten 
Gateway aus dem Bereich des Smart-Home verglichen.
Aufgrund der gewonnen Erkenntnisse bei der Integration und der Anfertigung der Teststellung werden zum Abschluss des Kapitels 
einige Empfehlungen zur Weiterentwicklung von OpenMUC gegeben.\\

Den Abschluss dieser Arbeit bildet \autoref{cha:discussion}. Innerhalb dieses Kapitels werden die Ergebnisse, hinsichtlich der Forschungsfragen, eingehend diskutiert.
Beendet wird das letzte Kapitel mit einem Ausblick auf weiterführende Forschungsfragen, deren Beantwortung innerhalb dieser Arbeit zu umfassend ist.\\

Im nächsten Kapitel werden, basierend auf der Recherche, verwandte Arbeiten vorgestellt und die Grundlagen für 
die weiterführenden Kapitel gelegt.



