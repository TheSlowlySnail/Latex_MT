\chapter{Diskussion der Ergebnisse}
\label{cha:discussion}

Dieses Kapitel dient der Erörterung der Ergebnisse der vorliegenden Arbeit.
Dazu werden die erzielten Ergebnisse in Bezug auf die eingangs formulierten Forschungsfragen diskutiert. % und es wird ein persönliches Fazit gezogen.
Den Abschluss bildet ein Ausblick auf weiterführende Forschungsfragen.

\section{Fazit}

Die Problemstellung zu Beginn, bestand im Fehlen eines Softwareverteilungsprozesses für das OpenMUC-Framework.
Das Fehlen dieses Prozesses zur Installation und Aktualisierung von Bundles erschwert nicht nur die Administration und Wartbarkeit einer OpenMUC-Installation,
sondern stellt im Falle der Aktualisierung auch ein Sicherheitsrisiko dar.
Außerdem wird der Einsatz von Softwarekomponenten Dritter beeinträchtigt,
was wiederum die Anwendbarkeit von OpenMUC in anderen Anwendungsszenarien und \ac{IoT}-Domänen einschränkt.
Aus diesem Grund wurden zu Beginn sechs Forschungsfragen aufgestellt, deren Beantwortung zur Lösung der Problemstellung notwendig ist.\\

\textit{Forschungsfrage 1} dient der Erhebung von Open-Source-Werkzeugen zur Softwareverteilung für \ac{OSGi}-basierte Anwendungen.
Dazu wurden im Rahmen einer Recherche die Werkzeuge Apache Felix OBR, Apache ACE, Apache Karaf und Eclipse hawkBit ermittelt und eingehend untersucht.
Aufgrund der Beschränkung auf Open-Source-Werkzeuge, lässt sich nicht ausschließen, dass im kommerziellen Sektor weitere geeignete Werkzeuge existieren.
Auch im Open-Source-Bereich sind, aufgrund der Ausrichtung ihrer Entwicklung, nicht alle verfügbaren Werkzeuge untersucht worden.
\textit{Forschungsfrage 2} zielt auf eine eingehende Untersuchung der ermittelten Werkzeuge ab.
Die in \autoref{werkzeuge} erläuterten Informationen sind das Ergebnis dieser Frage und bilden die Grundlage der späteren Bewertung der Werkzeuge.
Die Darstellung der Aspekte der vier Werkzeuge beschränkt sich lediglich auf einen Teil ihres jeweiligen Funktionsumfanges.
Es handelt sich hierbei nicht um eine vollständige Evaluation.
Der Fokus der Untersuchung liegt auf den Eigenschaften, die zur Umsetzung des Softwareverteilungsprozesses relevant sind.\\

Zur Evaluation der Werkzeuge werden begründbare Kriterien benötigt.
Die Beantwortung von \textit{Forschungsfrage 3} liefert Kriterien zur Bewertung der Eignung von Werkzeugen zur Softwareverteilung.
In \autoref{sec:openmuc_scenarios} sind fünf Kriterien, die aus den Anforderungen der Einsatzszenarien abgeleitet wurden, aufgezeigt.
Sicherheit ist ein Kriterium mit zentraler Bedeutung für alle Bereiche der Informationstechnologie.
Besonders im Bereich \ac{IoT} nimmt das Thema Sicherheit einen hohen Stellenwert ein.
Die hohe Anzahl an Kommunikationsteilnehmern und die Diversität im Bereich der Protokolle
der bereitgestellten Dienste und der Anwendungen im Internet der Dinge, bieten viele Angriffsvektoren.
Hossain et al. \cite{iot_security_issues} kommen in ihrer Analyse von Sicherheitsschwachstellen, Herausforderungen und offenen Problemen im \ac{IoT}-Bereich zu demselben Schluss.
Die Kriterien zentrales Managment, Erstellung von Distributionen, Ausfallsicherheit und Verfügbarkeit von Schnittstellen ergeben sich aus der Analyse der 
Anwendungsfälle von OpenMUC. Die drei Werkzeuge Apache ACE, Apache Karaf und Eclipse hawkBit erfüllen, mit einer Ausnahme, diese Kriterien.
Einzig Apache Karaf verfügt nicht über die Möglichkeit eines zentralen Managements.
Diese Tatsache zeigt, dass diese Eigenschaften von wesentlicher Bedeutung für Werkzeuge zur Verwaltung und Verteilung von Softwarekomponenten sind.
Die Kriterien existierende Community, erkennbare Weiterentwicklung, langlebiges Projekt, flexible Wartbarkeit
und die Anzahl der Entwickler sind der Arbeit von Cruz et al. \cite{evaluation_criteria} entnommen.
Es handelt sich hierbei um eine Auswahl von Kriterien und umfasst nicht alle in der Arbeit beschriebenen Merkmale.
In der Arbeit von Cruz et al. werden verschiedene Kriterien anhand von Szenarien, die unterschiedliche Einsatzzwecke der Software beschreiben, empfohlen.
Für die vorliegenden Einsatzzweck eines Werkzeuges zur Softwareverteilung exisitert kein passendes Szenario.
Aus diesem Grund sind Kriterien ausgewählt worden, welche die Langlebigkeit und Anpassbarkeit an eigene Anforderungen begünstigen.
Es kann angenommen werden, dass ein Werkzeug, das die Kriterien erfüllt, über einen längeren Zeitraum eingesetzt und an zukünftige Änderungen
des OpenMUC-Frameworks angepasst werden kann.\\

Zur Beantwortung von \textit{Forschungsfrage 4} werden die vier ausgewählten Werkzeuge hinsichtlich der definierten Kriterien untersucht und bewertet.
Das Ergebnis dieses Prozesses ist eine Entscheidungsmatrix, die in \autoref{tab:rating_matrix} abgebildet ist.
Die Bewertung der beiden best­plat­zierten Werkzeuge fällt denkbar knapp aus.
Dadurch stellt sich die Frage, ob die Auswahl von weiteren Kriterien, oder das Weglassen solcher, die Bewertung zu Gunsten von Eclipse hawkBit verändern würde.
Die prototypische Integration von Apache ACE in \autoref{sec:integration_ace} zeigt allerdings, dass das Werkzeug sehr gut in OpenMUC integriert werden kann.
Die notwendigen Anpassungen an OpenMUC für eine erste Integration fallen minimal aus.
Durch den Fokus auf die Softwareverteilung von \ac{OSGi}-Bundles bei der Entwicklung von Apace ACE, ist nur ein minimaler Entwicklungsaufwand nötig, um die Lücke 
im Softwareverteilungsprozess für OpenMUC zu schließen.
Da es sich bei Eclipse hawkBit um eine generische Lösung zur Softwareverteilung handelt, ist davon auszugehen, dass der Aufwand zur Integration deutlich höher ausfällt. 
Zur Anbindung an den bestehenden Softwareverteilungsprozess, müsste ein OpenMUC-Adapter für eine der Geräteschnittstellen
von Eclipse hawkBit entwickelt werden.
Eine Implementierung beider Werkzeuge und ein Vergleich dieser würde jedoch den Rahmen dieser Arbeit sprengen.\\

Hinsichtlich \textit{Forschungsfrage 5} können zwei Fälle unterschieden werden. 
Erstens die notwendigen Anpassungen von OpenMUC, die in \autoref{sec:integration_ace} beschrieben sind.
Die notwendigen Anpassungen umfassen eine Änderung der Konfiguration der Felix-Plattform und
die Bereitstellung eines Resource-Processors für nicht standardmäßig durch Apache ACE unterstützte Artefakte.
Dieser erste Prototyp vervollständigt zwar den Softwareverteilungsprozess, schöpft aber das mögliche Potenzial nicht aus.
Zweitens werden aus diesem Grund fünf Empfehlungen zur Weiterentwicklung des OpenMUC-Frameworks gegeben.
Deren Bearbeitung würde jedoch den Rahmen dieser Arbeit übersteigen.
Hervorzuheben ist hier der Einsatz von Deployment-Paketen und die Nutzung der Apache ACE-Template-Engine.
Die Umsetzung der Empfehlungen hilft dabei, weitere Teile der \ac{OSGi}-Plattform für OpenMUC zu nutzen
und die Funktionalität von Apache ACE nutzbringend einzusetzen.
Das führt zu einer besseren Benutzbarkeit und zu mehr Flexibilität beim Einsatz von Apache ACE.
Die notwendigen Anpassungen sind identifiziert und \textit{Forschungsfrage 5} damit hin­rei­chend
beantwortet. 
Es ist anzunehmen, dass eine gezielte Untersuchung des OpenMUC-Frameworks hinsichtlich des 
Verbesserungspotenzials, in Bezug auf die \ac{OSGi}-Plattform, weitere Möglichkeiten zur Anpassungen aufdeckt.\\

Anhand von \textit{Forschungsfrage 6} soll untersucht werden, ob das OpenMUC-Framework, das als Energy-Management-Gateway konzipiert wurde,
durch die Integration eines Softwareverteilungsprozesses mehr Flexibilität erlangt.
Die Erkenntnisse des Einsatzes von OpenMUC im Smart-Home-Labor der Hochschule Furtwangen fallen positiv aus.
Der Einsatz von OpenMUC innerhalb des Smart-Home zeigt, dass durch die bestehende Modularität und neu gewonnene Flexibilität 
auch Einsätze außerhalb des primären Einsatzbereiches möglich sind.
Im direkten Vergleich von OpenMUC und openHAB wird deutlich, dass beide Frameworks unterschiedliche Stärken und Schwächen haben,
jedoch beide ihre Aufgaben im Smart-Home erfüllen.
Natürlich handelt es sich bei der Teststellung nur um eine von vielen \ac{IoT}-Domänen. 
Sie liefert keine Erkenntnisse für weitere Domänen.
Die Einsatzfähigkeit von OpenMUC in allen Bereichen des Internets der Dinge kann damit nicht belegt werden.\\

Abschließend kann festgelegt werden, das mit der Integration von Apache ACE in das OpenMUC-Framework eine Lösung für die eingangs beschriebe Problemstellung gefunden wurde.
Innerhalb dieser Arbeit konnten jedoch nicht alle zutage getretenen Probleme eingehend untersucht werden.
Aus diesem Grund wird im Folgenden ein Ausblick auf mögliche weitere Forschungsthemen gegeben.

\section{Ausblick}

Im Rahmen dieser Arbeit sind verschiedene Fragen und Probleme zutage getreten, deren vollständige Beantwortung an dieser Stelle zu weitreichend wäre.
Sie können jedoch als Grundlage für weiterführende Untersuchungen genutzt werden.\\

Die Auswahl eines Werkzeuges zur Softwareverteilung in \autoref{sec:matrix} ging nur knapp zu Gunsten von Apache ACE aus.
Derzeit überzeugt Apache ACE durch seine Produktreife und seine Spezialisierung auf die \ac{OSGi}-Plattform.
Bei Eclipse hawkBit handelt es sich um ein recht junges Projekt, dem bisher wenig wissenschaftliche Aufmerksamkeit zuteil wurde.
Aus diesem Grund wäre in einigen Jahren eine erneute Untersuchung von Eclipse hawkBit in Bezug auf OpenMUC wünschenswert.
Außerdem könnte zum jetzigen Zeitpunkt eine allgemein verfasste Untersuchung des Werkzeuges zu interessanten Erkenntnissen aus dem Bereich der Softwareverteilung
im \ac{IoT}-Umfeld führen.\\

Das OpenMUC-Projekt weist einige Schwächen im Bereich seines Softwareentwicklungsprozesses auf.
Neben dem Problem der Versionierung, wie in \autoref{sec:recommendations} beschrieben, fehlt es dem Prozess 
an einer kontinuierlichen Integration (\ac{CI}) und einer kontinuierlichen Auslieferung (\ac{CD}) der Bundles.
Es besteht demnach Bedarf an der Evaluierung geeigneter \ac{CI}- und \ac{CD}-Technologien und
ihrer anschließenden Integration in das Betriebskonzept von OpenMUC.
Die Forschung in diesen Bereichen würde zu einer Steigerung der Professionalität in Bezug auf die Softwareentwicklung des Projektes führen.\\

Zu Beginn der Entwicklung von OpenMUC war das \ac{OSGi}-Release 4.2 (R4.2) die aktuellste Veröffentlichung der Spezifikation. 
In der Zwischenzeit sind drei weitere Versionen der \ac{OSGi}-Spezifikation veröffentlicht worden.
OpenMUC wurde in diesem Zeitraum ebenfalls stetig weiterentwickelt.
Das hat dazu geführt, dass die \ac{OSGi}-Plattform mittlerweile deutlich mehr Funktionalität aufweist und diese zum Teil vergleichbar mit der Funktionalität des OpenMUC-Frameworks ist.
Eine Analyse des OpenMUC-Frameworks im Hinblick auf den Funktionsumfang der aktuellen \ac{OSGi}-Plattform würde helfen, die Übereinstimmungen zu identifizieren
und die vergleichbaren Komponenten des Frameworks durch die der \ac{OSGi}-Plattform zu ersetzen.
Dies wiederum würde den zu pflegenden Quellcode von OpenMUC reduzieren und das Framework besser in die \ac{OSGi}-Plattform integrieren.\\

Die Frage nach dem Einsatz von OpenMUC in anderen \ac{IoT}-Domänen konnte im Rahmen dieser Arbeit nur für den Einsatz im 
Bereich des Smart-Home geklärt werden.
Eine Untersuchung zum Einsatz von OpenMUC in weiteren \ac{IoT}-Domänen wäre wünschenswert. 
Die daraus gewonnenen Erkenntnisse zu den Stärken und Schwächen des Frameworks, sowie den Anforderungen der verschiedenen Domänen,
können zur Verbesserung und zur Planung zukünftiger Features von OpenMUC genutzt werden.
Die kontinuierliche Verbesserung und Erweiterung von OpenMUC führt zu weiteren Anwendungsmöglichkeiten und einer Verbesserung der Akzeptanz des Frameworks.
Dies wiederum ist die Voraussetzungen für eine breite Nutzerschaft.