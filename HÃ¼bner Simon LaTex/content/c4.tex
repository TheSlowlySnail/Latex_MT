\chapter{Prototypische Integration von Apache ACE}
\label{cha:prototype}
In einem ersten Schritt soll Apache ACE zur Verwaltung der OpenMUC-Installationen im Smart-Energy-Labor des Fraunhofer \ac{ISE} eingesetzt werden.
Dabei sollen Probleme und Herausforderungen der Integration erkannt und erste Lösungsansätze erarbeitet werden.
In einem zweiten Schritt wird der Prototyp im Rahmen eines Szenarios des Smart-Home-Labor der Hochschule Furtwangen eingesetzt.
Dadurch sollen weitere Erkenntnisse in Bezug auf den Prototypen gewonnen werden. 
Im Weiteren werden die Möglichkeit zur Abbildung von Routinen im Smart-Home sowohl von OpenMUC, als auch von openHAB diskutiert.
Zum Abschluss dieses Kapitels werden die gewonnenen Erkenntnisse genutzt, um Vorschläge zur weiteren Entwicklung von OpenMUC zu unterbreiten. 

\section{Die Integration von Apache ACE in OpenMUC}
\label{sec:integration_ace}
% erkannte Probleme, Empfehlungen, 
Die Softwareverteilung über Apache ACE erfolgt zunächst in dem in \autoref{sec:openmuc_scenarios} beschriebenen Projekt CheapFlex.
Das Potential von CheapFlex wird unter anderem an prototypischen Implementierungen von Energie-Magement-Systemen innerhalb des Smart-Energy-Labor des Fraunhofer \ac{ISE} untersucht \cite{cheapflex}.
Im Rahmen dieses Projektes wird ebenfalls die Softwareverteilung für OpenMUC erprobt. Zu diesem Zweck wird zuerst ein Betriebskonzept für das Zusammenspiel zwischen Apache ACE und OpenMUC
entwickelt. Im Anschluss werden die Probleme und Herausforderungen, die bei der Integration identifiziert wurden, erläutert und Lösungsansätze präsentiert.
%Der Einsatz des neuen Betriebskonzeptes im Rahmen des Projektes CheapFlex führte zu weiteren Fragestellungen, welche am Ende dieses Abschnittes beantwortet werden.

\subsection{Das neue Betriebskonzept für OpenMUC}
\label{subsec:operational_concept}

Die Integration von Apache ACE erfordert ein neues Betriebskonzept für das gesamte OpenMUC-Projekt. 
Derzeit wird OpenMUC als Archiv bestehend aus dem Quellcode der Bundles, den kompilierten Bundles und dem vorkonfigurierten
Framework ausgeliefert.
Im neuen Betriebskonzept sollen die OpenMUC-Installationen des Fraunhofer ISE über einen Apache ACE-Server administriert werden.
Außerdem sollen andere Firmen und Institutionen die Möglichkeit haben, ihre Installationen über einen ACE-Management-Server zu verwalten.
Um diesen Anforderungen gerecht zu werden, wurde das in \autoref{fig:operational_concept_openmuc} zu sehende Betriebskonzept entworfen.
Ein \ac{CI}- und \ac{CD}-Server erzeugt bei einer Veröffentlichung das entsprechende \ac{OSGi}-Bundle und löst zwei weitere Schritte aus.

\begin{figure}[ht]
 \centering
 \includegraphics[width=\textwidth]{../02.Diagramme/operational_concept.png}
 \caption{Neues Betriebskonzept des OpenMUC-Projektes}
 \label{fig:operational_concept_openmuc}
\end{figure}

Im ersten Schritt wird das Bundle über die REST-Schnittstelle von Apache ACE in das interne OBR des zentralen ACE-Servers aufgenommen. 
Im zweiten Schritt wird das Bundle in einem frei zugänglichen \ac{OBR} veröffentlicht. 
Abhängig von der gewählten Konfiguration wird das neue Bundle direkt auf den Targets installiert oder solange zurückgehalten, bis ein 
Administrator das Update autorisiert.\\

Targets können über das lokale Netzwerk oder über das Internet mit dem ACE-Server am Fraunhofer ISE verbunden werden.
Bei Projekten mit einer größeren Anzahl an Targets empfiehlt sich der Einsatz eines Relay-Servers. 
In diesem Fall wird das Bundle an den Relay-Server übermittelt und dieser sorgt im folgenden für die Softwareverteilung.
Besonders in Fällen, in denen sich der Relay-Server und die Targets in einem gemeinsamen lokalen Netzwerk befinden, ist dieses Vorgehen effizient und empfehlenswert.
Auf diese Art und Weise muss das Bundle nur einmalig über das Internet übertragen werden und kann anschließend über das lokale Netzwerk ausgerollt werden.\\

Andere Firmen und Institutionen können das öffentliche \ac{OBR} in ihren eigenen ACE-Server integrieren und auf diesem Wege ihre OpenMUC-Installationen verwalten.
Die Bereitstellung des Frameworks als Archiv-Datei ist jedoch weiterhin notwendig.
So bleibt die Möglichkeit erhalten, das OpenMUC-Framework ohne die Notwendigkeit einer Apache ACE-Infrastruktur zu betreiben.\\

Nachdem das neue Betriebskonzept definiert ist und die Abläufe deutlich sind, kann die erste Installation ausgerollt werden.
Dabei wird deutlich, dass weitere Arbeit in Bezug auf die Integration zu leisten ist. 
Im nächsten Abschnitt werden die beiden Herausforderungen beschrieben und die erarbeiteten Lösungsansätze aufgezeigt.

\subsection{Herausforderungen bei der Integration von Apache ACE}

Das erste Problem besteht in einer Konfigurationsdirektive für die \ac{OSGi}-Plattform. 
Das OpenMUC-Projekt setzt auf die freie \ac{OSGi}-Implementierung Apache Felix\footnote{\url{http://felix.apache.org/}}.
Apache Felix kann über eine Konfigurationsdatei an unterschiedliche Anforderungen angepasst werden und unterstützt die von \ac{OSGi} spezifizierten Konfigurationsdirektiven.
Im Falle von OpenMUC ist die Direktive \textit{org.osgi.framework.storage.clean} auf den Wert \textit{onFirstInit} gesetzt.
Diese Direktive steuert, ob und wann der Speicherbereich des \ac{OSGi}-Frameworks aufgeräumt wird \cite[S. 99]{osgi_r6}.
Der Wert \textit{onFirstInit} sorgt dafür, dass der Speicherbereich des Frameworks vor der ersten Initialisierung des Framework-Bundles bereinigt wird \cite[S. 99]{osgi_r6}.\\

Das bedeutet, bei jedem Neustart des Apache Felix-Frameworks werden alle bereits installierten Bundles und ihre Daten aus dem Speicher gelöscht.
Im Falle von OpenMUC befinden sich die Bundles im \textit{bundle}-Verzeichnis, innerhalb des Arbeitsverzeichnisses.
Felix unterstützt ein automatisches Deployment von Bundles, die im \textit{bundle}-Verzeichnis abgelegt sind.
Diese beiden Automatismen sorgen dafür, dass bei einem Neustart alle Bundles gelöscht und neu installiert werden. 
Dieses Verhalten ist dann sinnvoll, wenn Bundles im \textit{bundle}-Verzeichnis aktualisiert werden und sichergestellt werden soll,
dass diese auch in der neusten Version installiert werden.\\

Dieses Verhalten führt jedoch zu einem Konflikt beim Einsatz von Apache ACE.
Die Installation von Bundles fällt in den Aufgabenbereich von Apache ACE, das die Bundles nach einer Installation nicht weiter vorhält.
Installierte Bundles existieren demnach nur im Speicherbereich des Frameworks. 
Wird dieser gelöscht, gehen alle Bundles des Targets verloren und müssen neu verteilt und installiert werden.
Dieses Verhalten sorgt im Zusammenspiel mit Apache ACE für einen unnötigen Einsatz von Ressourcen. 
Apache ACE sorgt zuverlässig dafür, dass aktualisierte Bundles auf den Targets eingespielt werden. Aus diesem Grund kann die 
Konfigurationsdirektive \textit{org.osgi.framework.storage.clean} an dieser Stelle auf den Wert \textit{none} geändert werden.
Das Setzen dieses Wertes sorgt dafür, dass die Konfigurationsdirektive nicht mehr ausgewertet wird \cite[S. 99]{osgi_r6}.\\

Die zweite Herausforderung besteht in der Verteilung von Artefakten, die nicht standardmäßig von Apache ACE unterstützt werden.
Für die Funktionalität von OpenMUC sind neben den \ac{OSGi}-Bundles weitere Dateien notwendig. 
Neben verschiedenen Konfigurationsdateien handelt es sich hierbei um Zertifikate zur Verschlüsselung der \linebreak \ac{HTTPS}-Verbindung und um Dateien für das Web-UI.
Apache ACE ist mit dem Ziel entwickelt worden, beliebige Artefakte zu verwalten und auf den Targets zu verteilen.
Unterstützt werden derzeit jedoch lediglich \ac{OSGi}-Bundles und XML-Dateien, die der 
Auto-Configuration-Spezifikation\footnote{OSGi Compendium Release 6, Kapitel 115 Auto Configuration Specification} von \ac{OSGi} genügen.
Für alle weiteren Artefakte, die mit Apache ACE verwaltet und verteilt werden sollen, muss die entsprechende Funktionalität bereitgestellt werden.\\

Zum einen muss für jeden neuen Artefakt-Typ ein Resource-Processor implementiert werden.
Der Resource-Processor ist ein Java-Interface, das im \ac{OSGi}-Compendium\footnote{OSGi Compendium Release 6, Kapitel 114, Abschnitt 10 Resource Processors} spezifiziert ist. 
Eine Implementierung stellt auf dem Target die notwendige Funktionalität zur Verarbeitung der spezifischen Artefakt-Typen bereit.
Zum anderen müssen zwei Dienste für Apache ACE implementiert werden.
Diese werden benötigt, um den Typ des Artefaktes und den dazugehörigen Resource-Processor zu ermitteln.
Um die Artefakte verteilen zu können, müssen zwei Vorrausetzungen erfüllt sein.
Erstens muss der Dienst, der das ArtifactHelper-Interface und ArtifactRecognizer-Interface implementiert, als Bundle auf dem Apache ACE-Server installiert sein.
Zweitens muss der Resource-Processor als Bundle im internen \ac{OBR} des Servers verfügbar sein.
Sind diese Voraussetzungen erfüllt, erfolgt die Verteilung der Artefakte, wie in \autoref{fig:resource_processor} dargestellt.

\begin{figure}[h]
 \centering
 \includegraphics[scale=0.23]{../02.Diagramme/resource_processor.png}
 \caption{Erkennung von Artefakten und deren Verarbeitung}
 \label{fig:resource_processor}
\end{figure}

Wird ein Artefakt hochgeladen (1), werden alle verfügbaren Dienste zur Erkennung von Artefakt-Typen (2) befragt. 
Erkennt ein Dienst das Artefakt, wird es zusammen mit dem korrespondierenden Resource-Processor-Bundle und
verschiedenen Meta-Daten wie dem Dateinamen und dem MIME-Type an das Target ausgeliefert (3).
Auf dem Target wird zuerst das Resource-Processor-Bundle installiert (4) und im Anschluss das Artefakt mit Hilfe des Resource-Processors verarbeitet (5).\\

Im Folgenden wird der durch Apache ACE vervollständigte Prozess zur Softwareverteilung vorgestellt.


\subsection{Vervollständigung des Softwareverteilungsprozesses}

Mit dem Einsatz von Apache ACE werden die von Arcangeli et al. \cite{sw_dist} definierten Aktivitäten, die in \autoref{prozess_der_softwareverteilung} beschrieben sind, vollständig abgebildet.
Wie in \autoref{fig:osgi_deploy_process_2} zu sehen, sind die Aktivitäten zwischen dem Apache ACE-Server und dem Apache ACE-Target (Client) aufgeteilt. 

\begin{figure}[h]
 \centering
 \includegraphics[width=11cm]{../02.Diagramme/osgi_deploy_process_2.png}
 \caption{OpenMUC Softwareverteilung auf Basis des Prozesses von Arcangeli et al. \cite{sw_dist}}
 \label{fig:osgi_deploy_process_2}
\end{figure}

Die verfügbaren Artefakte werden auf den Apache ACE-Server hochgeladen.
Dieser sorgt dafür, dass die benötigten Artefakte zum Target gelangen. 
Dort werden die Artefakte vom Apache ACE-Target über den Bundle-Context, wie in \autoref{subsec:bundle_lifecycle} beschrieben, installiert und gestartet.
Deinstallationen und Aktualisierungen werden vom Server initiiert und vom Target durchgeführt.
Die Reorganisation und Neuverteilung kann ebenfalls anhand von Distributionen abgebildet und durchgeführt werden.


\section{Teststellung im Smart-Home-Labor der Hochschule Furtwangen}
\label{sec:testing_shlab}

Neben der Erprobung im Projekt CheapFlex, kommt OpenMUC im Smart-Home Labor der Hochschule Furtwangen zum Einsatz.
Der Fokus der Teststellungen liegt dabei nicht nur auf dem Thema der Softwareverteilung.
Vielmehr geht es darum die, durch die Softwareverteilung neu gewonnene Flexibilität des Frameworks zu untersuchen und damit eine Antwort auf Forschungsfrage 6 
aus \autoref{sec:ziel} zu finden. Dazu wird eine typische Morgenroutine definiert und im Anschluss im Labor umgesetzt. 
Die erste Aufgabe besteht darin, OpenMUC in das Gesamtsystem des Labors zu integrieren. 
In einem weiteren Schritt wird die Morgenroutine spezifiziert und für das OpenMUC-Framework implementiert.
Zum Abschluss wird ein Vergleich zwischen OpenMUC und dem im Labor zum Einsatz kommenden openHAB durchgeführt.
Dazu ist es notwendig, die festgelegte Routine ebenfalls in openHAB zu implementieren.
Den Anfang macht die Integration von OpenMUC in das Smart-Home Labor der Hochschule Furtwangen.

\subsection{Integration von OpenMUC in das Smart-Home Labor}

Die Anbindung und Steuerung der Things im Smart-Home Labor erfolgt derzeit über openHAB.
Dabei handelt es sich um ein \ac{OSGi}-basiertes Gateway, das gezielt zur Erfüllung der Anforderungen im Smart-Home entwickelt wird.
Da im Labor die verschiedensten Anwendungsfälle und Routinen erprobt werden sollen, kommt eine Schichtenarchitektur zum Einsatz. 
Das ist notwendig, um zu verhindern, dass verschiedene Szenarien sich gegenseitig beeinflussen.
Der Aufbau des Labors, einschließlich der Integration von OpenMUC, ist in \autoref{fig:sh_lab_ex} zu sehen.
Die verschiedenen Bindings\footnote{Ein Binding verbindet die Things und andere Technologien mit openHAB \url{http://docs.openhab.org/addons/bindings.html}} 
erfolgen auf einem zentralen openHAB-Server.\\

\begin{figure}[h]
 \centering
 \includegraphics[scale=0.23]{../02.Diagramme/sh_lab_ex.png}
 \caption{Aufbau des Smart-Home-Labors der Hochschule Furtwangen, einschließlich der Integration von OpenMUC}
 \label{fig:sh_lab_ex}
\end{figure}

Der openHAB-Server stellt die verschiedenen Items\footnote{Ein Item stellt eine Eigenschaft und Fähigkeit des Smart-Home dar \url{http://docs.openhab.org/configuration/items.html}}
über das \ac{MQTT}-Protokoll zur Verfügung.
Das ermöglicht es, das Smart-Home über die \ac{MQTT}-Schnittstelle, den verschiedensten Anwendungen zur Verfügung zu stellen.
Die Regeln zur Steuerung werden auf den openHAB-Proxies realisiert. Zum jetzigen Zeitpunkt existiert für jedes Szenario ein eigener RaspberryPi mit einer openHAB Installation,
der mithilfe des \ac{MQTT}-Protokolls mit dem zentralen openHAB-Server verbunden ist. Dadurch lassen sich die verschiedenen Anwendungsfälle und Zuständigkeiten klar trennen.\\

Die Anbindung von OpenMUC erfolgt ebenfalls über \ac{MQTT}. Die Herausforderung beim Einsatz von
OpenMUC liegt in der Abstraktion der Things und ihrer Ansteuerung.
Zunächst wird die Morgenroutine vorgestellt. %, bevor das Thema näher vertieft werden kann.

\subsection{Die Morgenroutine}
\label{subsec:routine}

Das Smart-Home Labor umfasst vier Räume einer typischen Wohnung.
Neben dem Badezimmer gibt es eine Küche, ein Arbeitszimmer und einen Multimediaraum. % Sie alle sind mit für den Raum typischen Things ausgestattet.
Alle Räume sind mit allgemeinen Things, wie smarten Lampen und smarten Bewegungsmeldern ausgestattet. Hinzu kommen für den entsprechenden Raum typische 
Geräte wie eine smarte Kaffeemaschine in der Küche, oder ein kabelloses Soundsystem im Multimediaraum und dem Badezimmer.\\

Zum jetzigen Zeitpunkt befindet sich das Labor im Aufbau. Aus diesem Grund musste die anfänglich geplante Routine in ihrem Umfang reduziert werden.
Derzeit sind lediglich die Things im Multimediaraum und dem Badezimmer über \ac{MQTT} ansprechbar.
Die Routine selbst kombiniert den typischen Ablauf des morgendlichen Aufstehens mit den Möglichkeiten einer smarten Wohnung.
Dabei erstreckt sich der Ablauf über die beiden verfügbaren Räume.\\

Bevor die Routine startet, muss der Raum entsprechend präpariert werden.
Dafür werden die Lichter im Multimediaraum ausgeschaltet und die smarten Schiebetüren in den Modus opak versetzt.
Gestartet wird die Routine per Wecker\footnote{Das Wecker-Event wird momentan noch nicht über \ac{MQTT} publiziert. Deshalb wird das Event über ein Knopfdruck simuliert.},
der ein Event auf dem \ac{MQTT}-Bus veröffentlicht. Tritt das Event ein, wird das Abspielen einer vorprogrammierten Songliste veranlasst. 
Kurz nachdem das Wecker-Event eintritt, gehen die Lichter im Multimediaraum langsam an und verbreiten ein angenehmes warmes Licht.
Daraufhin werden die Lampen im Badezimmer auf ein gedimmtes warmes Licht eingestellt.
Gleichzeitig wird die Intensität des Lichtes im Multimediaraum erhöht und die Farbtemperatur kälter.
Betritt der Akteur das Badezimmer, werden die Lichter heller und die Musik wechselt vom Multimediaraum in das Badezimmer.
Nachdem der Akteur das Badezimmer verlässt, verstummt die Musik und er wird von einer freundlichen Stimme verabschiedet.
Kurz danach werden alle beteiligen Things in ihren ursprünglichen Zustand überführt und damit die Räume in einen neutralen Zustand versetzt.\\

Neben diesem Ablauf gibt es zwei funktionale Anforderungen an die Routine.
Zum einen muss die Geschwindigkeit der Ausführung regulierbar sein, um bei Demonstrationen einzelne Passagen zu beschleunigen.
Zum anderen muss die Routine jederzeit per Knopfdruck zu beenden sein.\\

Für die Umsetzung der Aufgabenstellung müssen zwei Systemkomponenten modelliert und umgesetzt werden.
Die erste Komponente bildet die Räume und Things als \ac{OSGi}-Dienste ab.
Der hier verfolgte Ansatz ermöglicht es, die Anforderungen an Modularität und Flexibilität aus dem Bereich des Smart-Home zu erfüllen.
Bei der zweiten Komponente handelt es sich um die Abbildung der eigentlichen Routine.

\subsection{Die Softwarearchitektur der Things in OpenMUC}
\label{subsec:sw_things}

Ziel ist es, jeden realen Raum und die dazugehörigen Things über einen wiederverwendbaren \ac{OSGi}-Service abzubilden.
Dazu wird für jedes Thing ein Interface, entsprechend seiner Funktionalität, definiert. 
Für die Benachrichtigung der Zustandsänderungen der Schalter und Bewegungsmelder, wird das Beobachter-Entwurfsmuster eingesetzt.
Dieses Muster definiert eine 1-zu-n-Verbindung zwischen Objekten und informiert abhängige Objekte über die Zustandsänderungen \cite{gof}.
Realisiert wird dieses Verhalten über zwei Interfaces. Eines für den Beobachter und eines für das zu beobachtende Thing (Subjekt).
Das vollständige Klassendiagramm ist in \autoref{fig:classes_things} dargestellt.\\

Die Abbildung der Räume erfolgt ebenfalls über ein Interface.
Um einen konkreten Raum zu repräsentieren, wird dieses Interface durch die Thing-Interfaces erweitert.
An dieser Stelle kommt das Fassade-Entwurfsmuster zum Einsatz. Dieses führt zu einer Schnittstelle, welche die Nutzung des Subsystems vereinfacht \cite{gof}.
In diesem Fall wird damit eine lose Kopplung zwischen den Routinen und den Things erreicht.
Eine Routine muss lediglich eine Referenz pro Raum vorhalten und nicht die Referenzen auf alle Things.
Außerdem benötigt die Routine somit keine detaillierten Kenntnisse der Things.
Die Operationen werden auf der Implementierung eines Raumes aufgerufen und an die Thing-Implementierung delegiert.
Innerhalb eines \ac{OSGi}-Bundles kann anhand des Interfaces und einer Implementierung ein Raum-Service instantiiert und verfügbar gemacht werden.
Im Folgenden wird ein generisches Softwaremodell entwickelt, anhand dessen verschiedene Routinen umgesetzt werden können.\\

\begin{figure}[H]
 \centering
 \includegraphics[scale=0.23]{../02.Diagramme/classes_things.png}
 \caption{Klassendiagramm der Things und Room-Services}
 \label{fig:classes_things}
\end{figure}

\subsection{Die Morgenroutine als Zustandsautomat}

Das Oxford English Dictionary definiert eine Routine wie folgt:
\glqq A sequence of actions regularly followed \grqq\ \cite{oed_routine}\\

Eine Routine beschreibt demnach eine Sequenz von Handlungen, die regelmäßig wiederholt wird.
Bezogen auf die Morgenroutine aus \autoref{subsec:routine} ergibt sich daraus eine Abfolge von Zuständen der Things im Smart-Home Labor.
Zur technischen Beschreibung und Spezifizierung der oben beschriebenen Routine bietet sich ein Zustandsautomat an. 
\autoref{fig:morning_routine_statemachine} zeigt die, in einen Zustandsautomaten überführte, Morgenroutine.

\begin{figure}[h]
 \centering
 \includegraphics[width=\textwidth]{../02.Diagramme/routines/morning_routine.png}
 \caption{UML-Zustandsdiagramm der Morgenroutine}
 \label{fig:morning_routine_statemachine}
\end{figure}

Die Routine kann wie gefordert jederzeit durch einen Knopfdruck in den \textit{reset}-Zustand überführt und damit beendet werden.
Nach dem Start befindet sie sich im Zustand \textit{prepared}. Bei Eintritt in diesen Zustand wird der Multimediaraum entsprechend vorbereitet.
Hier kann die Geschwindigkeit der Routine über einen Schalter festgelegt werden.
Über einen weiteren Knopfdruck oder das Alarm-Event des Weckers wird ein Zustandsübergang ausgelöst.
Im Multimediaraum geht ein warmes Licht an und die Routine befindet sich nun im Zustand \textit{awoken}.
Ein fünfminütiger Timer führt zum nächsten Zustandsübergang.
Beim Verlassen von \textit{awoken} wird das Licht im Multimediaraum kalt und von höherer Intensität.
Der Zustand \textit{bathroom prepared} veranlasst das Einschalten eines warmen, gedimmten Lichtes im Badezimmer.
Sobald der Akteur das Badezimmer betritt, wird in den Zustand \textit{bathroom entered} gewechselt und die Intensität des Lichtes erhöht.
Außerdem übernimmt das Soundsystem im Badezimmer die Musik aus dem Multimediaraum.
Verlässt der Akteur das Badezimmer, erfolgt der nächste Zustandsübergang. Nun befindet sich die Routine im Zustand \textit{bathroom left}.
Dort wird die Musik gestoppt und der Akteur wird von einer freundlichen Stimme verabschiedet.
Nach einer Minute aktiviert ein Timer den Zustandsübergang in den \textit{reset}-Zustand.
Dort werden die Räume in ihren ursprünglichen Zustand versetzt und die Routine endet.\\

Umgesetzt wurde die Routine unter Anwendungen des Zustands-Musters.
Das Klassendiagramm für der Routinen-Klassen ist in \autoref{fig:classes_routines} zu sehen.

\begin{figure}[H]
 \centering
 \includegraphics[scale=0.23]{../02.Diagramme/classes_routines.png}
 \caption{Klassendiagramm der Routinen}
 \label{fig:classes_routines}
\end{figure}

Das Zustands-Muster findet dann Anwendung, wenn sich die Verhaltensweise eines Objektes zur Laufzeit, abhängig von seinem Zustand, ändern muss \cite{gof}.
In unserem Fall ändert sich das Verhalten der \textit{entry}-, \textit{do}- und \textit{exit}-Methoden in Abhängigkeit zum momentan Zustand der Morgenroutine.
Für das Muster ist nicht festgelegt, ob die Routine oder der Zustand einen Zustandsübergang durchführt. 
In diesem Fall sind die Zustände verantwortlich für die Überführung in einen neuen Zustand.
Das ermöglicht den Einsatz einer generischen Routine-Klasse.
Diese muss initial mit einem konkreten Startzustand erzeugt werden und benötigt keine Kenntnisse über seine Zustände und Übergänge.
Die beiden Systemkomponenten zur Interaktion mit den Things und zur Abbildung von Routinen ermöglichen das Erstellen von beliebigen Routinen.
Abschließend werden die Möglichkeiten zur Implementierung von Routinen in openHAB und OpenMUC verglichen.

\section{openHAB und OpenMUC im Vergleich}

In openHAB werden Routinen über einen Satz von Rules\footnote{Regel zur Steuerung der Items: \url{http://docs.openhab.org/configuration/rules-dsl.html}}
abgebildet. Diese Rules unterscheiden sich deutlich von dem in OpenMUC verfolgten Ansatz.
Die Rules folgen dem Schema: Wenn \textit{(when)} Voraussetzung XY erfüllt, dann \textit{(then)} führe Aktion AB aus und beende \textit{(end)}.
Dabei unterliegt die Prüfung der Voraussetzung und die Ausführung von Aktionen weiteren Einschränkungen.
Zum Vergleich der beiden Systeme wurde die in \autoref{subsec:routine} beschriebene Routine ebenfalls mithilfe einer Rule in openHAB umgesetzt.
Im Folgenden werden die Vor- und Nachteile beider Umsetzungen anhand verschiedener Kriterien diskutiert.\\

\textit{Aufwand bei der Erstellung einer Routine}\\
Der Aufwand mit OpenMUC für code-basierte Routinen ist erwartungsgemäß hoch.
Für jeden Zustand einer Routine, muss eine eigene Klasse implementiert werden.
Außerdem muss die Routine als \ac{OSGi}-Bundle zur Verfügung stehen.
Allerdings sorgen die, bei dem Entwurf der beiden Systemkomponenten, gewählten Entwurfsmuster für eine deutliche Reduzierung des Aufwandes. 
Bei openHAB liegen die Regeln in Form einer Textdatei vor und müssen nicht kompiliert werden. 
Sie können durch ein Texteditor einfach erstellt und verändert werden.
Bei einfachen Abläufen ist der Aufwand dadurch minimal.
Werden die Routinen jedoch komplexer, steigt aufgrund der vorherrschenden Beschränkungen der Aufwand mit openHAB an.
Zum Beispiel sind die Voraussetzungen für die Abarbeitung einer Regel nicht frei wählbar. 
Es kann nur auf die Zustandsänderung eines Items, auf ein zeitliches Ereignis, oder auf ein System-Event reagiert werden.
Will man die Abarbeitung einer Rule abhängig vom Zustand der Routine machen, muss dafür ein virtuelles Item in openHAB angelegt werden.
Dieses Item repräsentiert dann den aktuellen Zustand der Routine und kann als Voraussetzung für eine Änderung genutzt werden.
Es bleibt festzuhalten, dass der Aufwand zur Erstellung von Routinen in openHAB auf Kosten von Flexibilität bei der 
Regelerstellung geringer ausfällt, als bei OpenMUC.\\

\textit{Einbindung und Ansteuerung von Things}\\
An dieser Stelle profitiert OpenMUC von der \ac{MQTT}-Schnittstelle.
Die Items, die über den Bus zur Verfügung gestellt werden können über den Data-Manager in OpenMUC abgebildet werden.
Sie können von den Implementierungen aus \autoref{subsec:sw_things} genutzt werden.
Neue Things erfordern, dass neue Interfaces definiert werden und die entsprechenden Räume angepasst werden.
Im Anschluss muss das \ac{OSGi}-Bundle in einer neuen Version gebaut und verteilt werden.
Die Verteilung kann zwar durch den Einsatz von Apache ACE automatisiert erfolgen, aber der Aufwand für die Aktualisierung der Softwarekomponente bleibt.
Die openHAB-Proxies setzen ebenfalls auf die \ac{MQTT}-Schnittstelle. Allerdings werden die verfügbaren Items in einer Textdatei definiert. 
Diese kann auf dem zentralen openHAB-Server erzeugt und an die Proxies verteilt werden.
Der Nachteil ist, dass die Verteilung manuell erfolgen muss und nicht über ein Werkzeug wie Apache ACE automatisiert erfolgt. 
Bei einer großen Anzahl an Proxies besteht die Schwierigkeit darin, alle Proxies zuverlässig auf demselben Stand zu halten.\\

\textit{Nötige (Fach)Kenntnisse}\\
Die notwendigen Fachkenntnisse sind bei OpenMUC deutlich höher anzusetzen.
Es werden fortgeschrittene Kenntnisse der Themen objektorientierte Programmierung, Java-Programmierung und OSGi benötigt.
Das ist eine klare Einschränkungen bei der Nutzung von OpenMUC.
Die Zielgruppe von openHAB ist eine andere.
Der einfache Aufbau von Rules erlaubt es Menschen, ohne große Kenntnisse über Programmierung, einfache Steuerungen für das Smart-Home zu entwickeln.
Allerdings werden mit steigender Komplexität der Routinen auch hier Experten benötigt.
Zum Beispiel, um nicht im Standard enthaltene Aktionen zu implementieren.\\

\textit{Voraussetzungen und Aktionen}\\
In diesen beiden Punkten unterliegt openHAB, im Vergleich zu OpenMUC, starken Einschränkungen.
Die Voraussetzung für den Start einer Regel ist auf die Zustandsänderung von Items, den Eintritt eines Zeitpunktes und den Zustand des Systems beschränkt.
Zum Beispiel muss das fünfminütige Warten, das den \textit{awoken}-Zustand in den \textit{bathroom prepared}-Zustand überführt,
über einen Timer\footnote{Ein Timer kann innerhalb eines Aktionsblockes erstellt werden und löst bei Ablauf weitere Aktionen aus.
Siehe \url{https://github.com/openhab/openhab1-addons/wiki/Actions\#timers}} und ein virtuelles Item abgebildet werden. 
Auch die Möglichkeiten für Aktionen innerhalb einer Rule sind begrenzt. Es gibt eine Reihe von Aktionen, 
die immer verfügbar sind und Aktionen, die durch verschiedene Bindings verfügbar werden.
Es ist möglich eigene Add-ons für Aktionen zu implementieren, das ist allerdings mit entsprechendem Aufwand verbunden.
Beim Einsatz von OpenMUC gibt es keine Einschränkungen für die Bedingung der Zustandsübergänge, oder Aktionen.
Es können verschiedene externe Datenquellen wie Wetterprognosen oder DeepLearning-Analysen als Bedingung genutzt werden.
Zur Realisierung von Aktionen, kann das gesamte Java-Ökosystem genutzt werden.
OpenMUC bietet an dieser Stelle deutlich mehr Freiheiten, als openHAB.
Allerdings auf Kosten der Komplexität und der erforderlichen Tiefe der Kenntnisse.\\

\textit{Abschließendes Fazit des Vergleiches}\\
OpenMUC und openHAB sind für unterschiedliche Zielgruppen und Anforderungen entwickelt.
Die Stärken von openHAB liegen in der einfachen Integration und Steuerung von Things.
Des Weiteren können, ohne tiefe Fachkenntnisse, verschiedene Szenarien für das Smart-Home entwickelt werden.
Die Einschränkungen von openHAB betreffen vermutlich nur einen kleinen Teil der Regeln bei komplexeren Routinen.
In diesen Fällen kommen die Stärken von OpenMUC zum Tragen. 
Bei der Abbildung von Szenarien und Routinen gibt es kaum Einschränkungen, allerdings sind diese Freiheiten mit einem erhöhten Aufwand verbunden.
Diese Freiheit ermöglicht den Einsatz von externen Bibliotheken und Datenquellen und erweitert damit die Funktionalität beträchtlich.
Dadurch steigt jedoch die erforderliche Fachkenntnis.\\
Da OpenMUC und openHAB mit unterschiedlichen Zielen entwickelt wurden, lässt sich aus deren Vergleich kein klarer Sieger ausmachen.
Die beiden Frameworks müssen immer auf den konkreten Anwendungsfall hin betrachtet werden.

Im Falle des Smart-Home-Labors der Hochschule Furtwangen würde ich den Einsatz von OpenMUC zur Umsetzung von Routinen aus folgenden vier Gründen bevorzugen:
\begin{enumerate}
 \item Bei den Nutzern des Labores ist die notwendige Fachkenntnis vorhanden.
 \item Die Kenntnisse der Nutzer im Bereich der objektorientierten Java-Programmie\-rung und der Softwarearchitektur werden weiter vertieft.
 \item Die Einbindung externer Bibliotheken und Datenquellen ermöglicht die Erstellung und Erprobung von komplexen Routinen.
 \item Die openHAB-Proxies können durch eine OpenMUC-Installation abgelöst werden. Die Routinen können per Apache ACE je nach Bedarf installiert und deinstalliert werden. 
\end{enumerate}

Abschließend werden basierend auf den gewonnen Erkenntnissen und Informationen Empfehlungen zur Weiterentwicklung des OpenMUC-Frameworks ausgesprochen.

\section{Empfehlungen zur Weiterentwicklung des OpenMUC-Frameworks}
\label{sec:recommendations}

Im Rahmen dieser Arbeit konnten einige Aspekte von OpenMUC identifiziert werden,
die durch Komponenten der \ac{OSGi}-Plattform ersetzt oder optimiert werden können.
Darüber hinaus verfügt Apache ACE über weitere Funktionalität, die durch das OpenMUC-Framework genutzt werden kann.
Allerdings sind dafür strukturelle Änderungen des OpenMUC-Frameworks erforderlich.
Im Folgenden werden die fünf Empfehlungen zur Weiterentwicklung des Frameworks vorgestellt.

\begin{enumerate}
 \item \textit{Deployment-Pakete:}
 Ein Deployment-Paket besteht aus einer Gruppe von Ressourcen. Eine solche Gruppe wird als gemeinsame Einheit angesehen \cite[S.391]{osgi_r6_compendium}.
 Die Ressourcen können  zum Beispiel Bundles, Konfigurationsdateien, oder Zertifikate sein.
 Das Deployment-Paket wird als normale \ac{JAR}-Datei, die auf den Suffix .dp endet, ausgeliefert \cite[S.393]{osgi_r6_compendium}.
 Installiert und deinstalliert wird es als atomare Einheit. 
 Im Falle von OpenMUC lässt sich damit die Verteilung von Artefakten, die nicht standardmäßig von Apache ACE unterstützt werden, vereinfachen.
 Anstatt die Konfigurationsdateien als Artefakte in Apache ACE vorzuhalten, können die entsprechenden Bundles zusammen mit ihren Ressourcen
 als Deployment-Paket ausgeliefert werden.
 
 \item \textit{Apache ACE-Template-Engine:}
 Im Falle von Konfigurationsdateien gibt es eine Alternative zu den Deployment-Paketen.
 \ac{OSGi} ermöglicht die Konfiguration von Bundles über ein definiertes XML-Format\footnote{OSGi Compendium Release 6, Kaptiel 105 Metatype Service Specification}. 
 Diese XML-Dateien können im Zusammenspiel mit der Template-Engine von Apache ACE genutzt werden.
 Über Apache ACE ist es möglich, Tags für Artefakte, Features, Distributionen und Targets festzulegen. Bei einem Tag handelt es sich um ein Schlüssel-Wert-Paar.
 Liegt zum Beispiel die Konfiguration des Web-UI als entsprechendes Template vor, kann der Port des \ac{HTTP}-Servers
 durch den Platzhalter \textit{\$\{content.webui\_port\}}\footnote{Jeder Tag im Template muss mit \textit{content.} eingeleitet werden. \url{https://ace.apache.org/docs/user-guide.html}} ersetzt werden.
 Für ein Target kann der Tag \textit{webui\_port} dann auf den Wert 4711 gesetzt werden. 
 Apache ACE sorgt bei der Auslieferung des Konfigurations-Templates dafür, dass für das entsprechende Target der Platzhalter ersetzt wird. 
 Wird das Tag hingegen an einer Distribution angebracht, wird der Platzhalter bei allen Targets, auf denen diese Distribution installiert wird, ersetzt.
 Mit Templates und Tags können große Gruppen von Targets bis hin zu einzelnen Targets konfiguriert werden.
  
 \item \textit{Sicherheit:}
 Wie in \autoref{sicherheit} dargelegt, geht von schadhaften Bundles ein großes Gefährdungspotenzial aus.
 Vor allem in Bezug auf die Einsatzszenarien sollte das Gefährdungspotenzial für OpenMUC so gering wie möglich gehalten werden.
 Die Fraunhofer-Gesellschaft betreibt eine eigene \ac{PKI}\footnote{Das Fraunhofer PKI Competence Center \url{https://www.pki.fraunhofer.de}},
 die zur digitalen Signatur der \ac{OSGi}-Bundles genutzt werden kann.
 Zur Erhöhung der Sicherheit sollten zum einen in Zukunft alle OpenMUC-Bundles signiert werden und
 zum anderen die Installation auf signierte Bundles beschränkt sein.
 Des Weiteren wird empfohlen, die Kommunikation des Apache ACE-Servers über die 2-Wege-SSL-Authentifizierung abzusichern.
 Durch diese zwei Maßnahmen kann die Gefährdung, die durch die beiden Angriffsvektoren Kompromittierung von Bundles und \ac{MITM}-Angriffe entsteht, reduziert werden.
   
 \item \textit{Nutzung des OBR für Einzelinstallationen:}
 Nicht in jedem Fall ist der Betrieb einer Apache ACE-Infrastruktur sinnvoll oder wünschenswert. 
 Für diesen Fall sollte der Prozess zur Installation und Aktualisierung von Bundles optimiert werden.
 Durch die Einbindung des Felix-OBR-Clients in OpenMUC kann das, im Betriebskonzept (siehe \autoref{subsec:operational_concept}) beschriebene, 
 öffentliche \ac{OBR} genutzt werden.
 Die Web-UI kann um die Funktionalität zur Konfiguration der zu befragenden Repositories und zur Installation und Aktualisierung der Bundles erweitert werden.
 Auf diesem Wege profitieren OpenMUC-Einzelinstallationen von der neuen Infrastruktur.
 Außerdem erhalten Anwender ein Werkzeug zur Installation und Aktualisierung von Bundles.
 
 \item \textit{Versionierung:}
 Die Bundles des OpenMUC-Frameworks werden derzeit mit dem Build-Tool Gradle\footnote{Projektwebseite Gradle: \url{https://gradle.org/}} erzeugt.
 Aufgrund der Abhängigkeiten zwischen den Bundles wird das gesamte Framework als Gradle-Multiproject verwaltet. 
 Ein Problem dabei ist, dass die einzelnen Unterprojekte nicht über eigenständige Versionsnummern verfügen.
 Gibt es innerhalb des Frameworks eine Änderung, die veröffentlicht werden soll, wird die Versionsnummer des Projektes hochgezählt.
 Das hat zur Folge, dass alle Bundles in einer neuen Version vorliegen. 
 Werden diese Bundles in das \ac{OBR} aufgenommen, werden über Apache ACE alle Bundles der zu verwaltenden Targets aktualisiert.
 Dieses Verhalten bindet zum einen unnötig Ressourcen und zum anderen kann es für Targets,
 die über das Mobilfunknetz mit dem Apace ACE-Server verbunden sind, zu hohen Kosten führen.
 Aus diesen Gründen ist es anzuraten, die momentane Konfiguration des Gradle-Projektes dahingehend zu überarbeiten,
 dass für jedes Bundle eine individuelle Versionsnummer vergeben wird.
 
\end{enumerate}

Das nächste Kapitel bildet den Abschluss dieser Arbeit. 
Dort werden die Ergebnisse diskutiert und ein Ausblick auf weiterführende Forschungsfragen gegeben.